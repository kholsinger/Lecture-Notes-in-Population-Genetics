\documentclass[titlepage,landscape]{seminar}
\usepackage{url}
\usepackage{graphicx}
\usepackage{hyperref}
\usepackage{epstopdf}
\usepackage{slides}

\newcommand{\frack}{\frac{1}{k}}
\newcommand{\quarter}{\frac{1}{4}}

\begin{document}

\myslide{
  \heading{Properties of genetic drift}

  \begin{itemize}

    \item We can predict the probability of an outcome, but we cannot
      say which of the possible outcomes will occur.

    \item The probability that an allele is fixed is equal to its
      current frequency.

    \item There is no directional tendency in allele frequency
      change. 

    \item The rate of drift is inversely proportional to the
      population size.

    \item The amount of uncertainty about allele frequencies in the
      next generation is inversely proportional to the population
      size.

    \item The rate at which a population accumulates identity by
      descent is inversely proportional to the population size.
      
  \end{itemize}
}

\myslide{
\heading{Genetic drift: Variance effective size}
\[
Var(p_{t+1}) = \frac{p_t(1-p_t)}{2N} \quad.
\]
\vfil
\[
N_e^{(v)} = \frac{p(1-p)}{2\widehat{Var}(p)}
\]
}

\myslide{
\heading{Genetic drift: Inbreeding effective size}
\begin{eqnarray*}
f_{t+1} &=& \frac{1}{2N} \quad \mbox{assuming $f_t = 0$} \\
N_e^{(f)} &=& \frac{1}{2\hat f_{t+1}}
\end{eqnarray*}
}

\myslide{
  \heading{Properties of genetic drift}

  \begin{itemize}

    \item Populations will tend to diverge if $4N_e(m + \mu) < 1$ and
      they will tend to remain similar if $4N_e(m + \mu) > 1$

    \item An allele that is selectively favored may be lost as a
      result of genetic drift.

      \[
        \mbox{P}(\mbox{newly arisen allele is fixed}) = 2s
      \]

    \item An allele that is selectively disfavored may be fixed as a
      result of drift

    \item Drift dominates the dynamics if $2N_es < 1$.
  \end{itemize}
}

\myslide{
  \heading{Kingman's coalescent}
\begin{center}
\resizebox{\textwidth}{!}{\includegraphics{coalescent-figure.eps}}
\end{center}
}

\myslide{
\heading{Kingman's coalescent (two allele copies)}

\[
\mbox{P}(\hbox{two alleles ibd}) = \frac{1}{2N_e^{(f)}}
\]
\vfill
}

\myslide{
\heading{Kingman's coalescent (two allele copies)}

\[
\mbox{P}(\hbox{two alleles ibd}) = \frac{1}{2N_e^{(f)}}
\]
\vfill
\[
\mbox{P}(\hbox{coalescent at time $t$} = \left(1 - \frac{1}{2N_e}\right)^{t-1}\frac{1}{2N_e}
\]
\vfill
}

\myslide{
\heading{Kingman's coalescent (two allele copies)}

\[
\mbox{P}(\hbox{two alleles ibd}) = \frac{1}{2N_e^{(f)}}
\]
\vfill
\[
\mbox{P}(\hbox{coalescent at time $t$} = \left(1 - \frac{1}{2N_e}\right)^{t-1}\frac{1}{2N_e}
\]
\vfill
\[
\hbox{Average time to coalescence of two alleles} = 2N_e
\]
}

\myslide{
\heading{Kingman's coalescent (multiple allele copies)}

\begin{eqnarray*}
m &=& \hbox{number of allele copies} \\
\hbox{number of allele copy pairs} &=& \frac{m(m-1)}{2} \\
\mbox{P}(\hbox{coalescent involving one pair of alleles}) &=& 
\left(\frac{m(m-1)}{2}\right)\left(\frac{1}{2N_e}\right) \\
\hbox{Average time to coalescent event} &=&
\left(\frac{2}{m(m-1)}\right)\left(2N_e\right) \\
&=& \frac{4N_e}{m(m-1)} 
\end{eqnarray*}
}

\myslide{
\heading{Kingman's coalescent (multiple allele copies)}

\begin{eqnarray*}
\hbox{Average time to coalescent event} &=& \frac{4N_e}{m(m-1)} 
\end{eqnarray*}
Now there are $m-1$ alleles
\begin{eqnarray*}
\hbox{Average time to next coalescent event} &=& \frac{4N_e}{(m-1)(m-2)} 
\end{eqnarray*}
\vfill
\begin{eqnarray*}
\bar t &=& \sum_{k=2}^m \frac{4N_e}{k(k-1)} \\
       && \mbox{TAMO} \\
       &=& 4N_e\left(1 - \frac{1}{m}\right) \\
       &\approx& 4N_e
\end{eqnarray*}
}

\myslide{
  \heading{Coalescent in continuous time}

  {\tt ms()} uses a continuous time version of the coalescent.

  Remember that
  \[
    \log(1-p) \approx -p \quad .
  \]
  \vfill
}

\myslide{
  \heading{Coalescent in continuous time}

  {\tt ms()} uses a continuous time version of the coalescent.

  Remember that
  \[
    \log(1-p) \approx -p \quad .
  \]
  \vfill
  Implying that
  \begin{eqnarray*}
    (1 - p)^t &=& e^{tlog(1-p)} \\
    &=& e^{-tp} \quad .
  \end{eqnarray*}
  \vfill
}

\myslide{
  \heading{Coalescent in continuous time}

  {\tt ms()} uses a continuous time version of the coalescent.

  Remember that
  \[
    \log(1-p) \approx -p \quad .
  \]
  \vfill
  Implying that
  \begin{eqnarray*}
    (1 - p)^{t-1} &=& e^{(t-1)log(1-p)} \\
    &=& e^{-(t-1)p} \quad .
  \end{eqnarray*}
  \vfill
  In our case
  \begin{eqnarray*}
    P(T_k = t) &=& \left(\frac{k(k-1)}{4N_e}\right)
                   \left(1 - \frac{k(k-1)}{4N_e}\right)^t \\
               &\approx& \left(\frac{k(k-1)}{4N_e}\right)
                         e^{-t\left(\frac{k(k-1)}{4N_e}\right)}
                         \quad .
  \end{eqnarray*}
}

\myslide{
\heading{Mitochondrial Eve}

Data
\begin{itemize}

\item Mitochondrial DNA from 147 humans

\item Most recent common ancestor 200,000 years ago

\end{itemize}

Expectation
\begin{itemize}

\item All mitochondrial genomes share a single common ancestor $2N_e$
  generations ago.

\item Human generation time $\approx$ 20 years. 

\item 200,000 years $\approx$ 10,000 generations.

\item Coalescence of all mitochondria consistent with $N_e$ of 5000.

\end{itemize}
}

\myslide{
\heading{Coalescent in structured populations}

{\small
\begin{eqnarray*}
\bar t_0 &=& \mbox{average time to coalescence of allele copies from
             the same population} \\
\bar t_1 &=& \mbox{average time to coalescence of allele copies from
             different populations} \\
\bar t &=& \mbox{average time to coalescence of allele copies} \\
       &=& \frac{k(k-1)\bar t_1 + k\bar t_0}{k^2}
\end{eqnarray*}
}
\vfill
If mutation is rare,
\[
F_{st} = \frac{\bar t - \bar t_0}{\bar t} \quad .
\]
}

\myslide{
  \heading{The coalescent and natural selection}

  \begin{itemize}

  \item Coalescent events happen only {\it within\/} allele classes.

  \item All allele copies within an allele class are selectively
    equivalent.

  \item The {\it within\/} allele coalescent process is neutral,
    i.e., the one we've already seen.

    \begin{itemize}

      \item If the effective size of the population is $N_e$ and the
        current allele frequency of allele $k$ is $p_k$, the
        probability of a coalescent event is

\[
 \frac{\frac{k(k-1)}{2}}{2N_ep_t} \quad .
 \]

    \end{itemize}

  \item The {\it between\/} allele class frequency changes according
    to ``classical'' selection and drift.  

  \end{itemize}
}


\myslide{
  \heading{The coalescent and natural selection}

  Here's how we simulate a sample from the coalescent with selection:

  \begin{enumerate}

  \item Simulate the trajectory of frequency changes among allele
    classes.

  \item Simulate the coalescent history within each allele class.

  \end{enumerate}

  The same approach can be used for any {\it structured\/} coalescent
  process, i.e., one with distinct allele classes where it is possible
  to simulate the trajectory of class frequencies.
  
}

\end{document}

