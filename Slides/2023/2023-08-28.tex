\documentclass[titlepage,landscape]{seminar}
\usepackage{url}
\usepackage{graphicx}
\usepackage{hyperref}
\usepackage{epstopdf}
\usepackage{slides}

\begin{document}

\myslide{
\heading{Mother-offspring combinations}
\begin{center}
\begin{tabular}{lrrr}
\hline\hline
                  & \multicolumn{3}{c}{Genotype of offspring} \\
Maternal genotype & $A_1A_1$ & $A_1A_2$ & $A_2A_2$ \\
\hline
$A_1A_1$          &      305 &      516 & \\
$A_1A_2$          &      459 &     1360 & 877 \\
$A_2A_2$          &          &      877 & 1541 \\
\hline
\end{tabular}
\end{center}

}

\myslide{
  \heading{Paternal allele frequencies}

\begin{enumerate}

\item 305 out of 821 male gametes that fertilized eggs from $A_1A_1$
mothers carried the $A_1$ allele (37\%).

\item 877 out of 2418 male gametes that fertilized eggs from $A_2A_2$
mothers carried the $A_1$ allele (36\%).

\end{enumerate}

How many of the 2696 male gametes that fertilized eggs from $A_1A_2$
mothers carried the $A_1$ allele?

}

\myslide {
\heading{Genotypes and allele frequencies}
\begin{center}
\begin{tabular}{ccr}
\hline\hline
Genotype & Number & Sex \\
\hline
$A_1A_1$ & $F_{11}$ & female \\
$A_1A_2$ & $F_{12}$ & female \\
$A_2A_2$ & $F_{22}$ & female \\
$A_1A_1$ & $M_{11}$ & male \\
$A_1A_2$ & $M_{12}$ & male \\
$A_2A_2$ & $M_{22}$ & male \\
\hline
\end{tabular}
\end{center}

$$\begin{array}{cc}
p_f = \frac{2F_{11}+F_{12}}{2F_{11}+2F_{12}+2F_{22}} &
q_f = \frac{2F_{22}+F_{12}}{2F_{11}+2F_{12}+2F_{22}} \\
 & \\
p_m = \frac{2M_{11}+M_{12}}{2M_{11}+2M_{12}+2M_{22}} &
q_m = \frac{2M_{22}+M_{12}}{2M_{11}+2M_{12}+2M_{22}}
\end{array}$$

$$p = \frac{1}{2}(p_f+p_m)$$
}

\myslide{
\heading{Maternal genotypes and allele frequencies}
$$\begin{array}{lll}
F_{11} &= 305 + 516 &= 821 \\
F_{12} &= 459 + 1360 + 877 &= 2696 \\
F_{22} &= 877 + 1541 &= 2418
\end{array}$$
\vfil
$$\begin{array}{lll}
p_f &= \frac{2(821) + 2696}{2(821) + 2(2696) + 2(2418)} &= 0.37 \\
q_f &= \frac{2(2418) + 2696}{2(821) + 2(2696) + 2(2418)} &= 0.63
\end{array}$$
Let's define
\begin{eqnarray*}
F_1 &=& 2F_{11} + F_{12} \\
F_2 &=& 2F_{22} + F_{12} \\
F   &=& 2F_{11} + 2F_{12} + 2F_{22}
\end{eqnarray*}
Then $p_f = F_1/F$ and $q_f = F_2/F$.
}

\myslide{
  \heading{Zero force laws}

  {\bf First Law of Population Genetics}

  If all genotypes at a particular locus have the same average
  fecundity and the same average chance of being included in the
  breeding population, allele frequencies in the population will
  remain constant from one generation to the next.

  Important because it tells us that if allele frequencies change from
  one generation to the next, either genotypes differ in fecundity or
  in their chances of being included in the breeding population.
}

\myslide{
\heading{Genotype vs. allele frequencies}
\begin{center}
\begin{tabular}{lrrr}
\hline\hline
& $A_1A_1$ & $A_1A_2$ & $A_2A_2$ \\
\hline
Population 1 & 50 & 0 & 50 \\
Population 2 & 25 & 50 & 25 \\
\hline
\end{tabular}
\end{center}

\begin{eqnarray*}
p_1 &=& (2*50)/(2*100) \\
    &=& 0.5 \\
p_2 &=& (2*25 + 50)/(2*100) \\
    &=& 0.5
\end{eqnarray*}
}

\myslide{
\heading{Mating table}
\begin{center}
\begin{tabular}{rcccc}
\hline\hline
                       &           & \multicolumn{3}{c}{Offsrping genotype} \\
Mating                 & Frequency     & $A_1A_1$ & $A_1A_2$ & $A_2A_2$ \\
\hline
$A_1A_1 \times A_1A_1$ & $x_{11}^2$     &        1 &        0 &        0 \\
              $A_1A_2$ & $x_{11}x_{12}$ &    $\half$ &    $\half$ &        0 \\
              $A_2A_2$ & $x_{11}x_{22}$ &        0 &        1 &        0 \\
$A_1A_2 \times A_1A_1$ & $x_{12}x_{11}$ &    $\half$ &    $\half$ &        0 \\
              $A_1A_2$ & $x_{12}^2$     &  $\fourth$ &    $\half$ &  $\fourth$ \\
              $A_2A_2$ & $x_{12}x_{22}$ &        0 &    $\half$ &    $\half$ \\
$A_2A_2 \times A_1A_1$ & $x_{22}x_{11}$ &        0 &        1 &        0 \\
              $A_1A_2$ & $x_{22}x_{12}$ &        0 &    $\half$ &    $\half$ \\
              $A_2A_2$ & $x_{22}^2$     &        0 &         0 &
                       1 \\
\hline
\end{tabular}
\end{center}
}

\myslide{
\heading{Offspring genotype frequencies}
\begin{eqnarray*}
\hbox{freq.}(A_1A_1\hbox{ in zygotes}) &=&
   x_{11}^2 + \frac{1}{2}x_{11}x_{12} + \frac{1}{2}x_{12}x_{11}
   + \frac{1}{4}x_{12}^2 \\
&=& x_{11}^2 + x_{11}x_{12} + \frac{1}{4}x_{12}^2 \\
&=& (x_{11} + x_{12}/2)^2 \\
&=& p^2 \\
\hbox{freq.}(A_1A_2\hbox{ in zygotes}) &=& 2pq \\
\hbox{freq.}(A_2A_2\hbox{ in zygotes}) &=& q^2 \\
\end{eqnarray*}
}

\myslide{
\heading{Binomial probability}
\[
\mbox{P}(K=k|p) = {N \choose k}p^k(1-p)^{N-k}
\]
Maximum likelihood estimate of $p$
\[
\hat p = \frac{k}{N}
\]
}

\myslide{
\vfil
\heading{Bayes' Theorem}
\begin{eqnarray*}
\mbox{P}(\phi|x) &=&
                     \frac{\mbox{P}(x|\phi)\mbox{P}(\phi)}{\mbox{P}(x)} \\
\\
\mbox{P}(x|\phi) &=& \mbox{likelihood} \\
\mbox{P}(\phi) &=& \mbox{prior distribution} \\
\mbox{P}(\phi|x) &=& \mbox{posterior distribution}
\end{eqnarray*}
\vfil
\[
\mbox{P}(p_f|F_1,F) = \frac{\mbox{P}(F_1|p_f,F)\mbox{P}(p_f)}{\mbox{P}(F_1)}
\]
}

\myslide{
\heading{Genotypes and phenotypes in the ABO blood group}
\begin{center}
\begin{tabular}{l|r|r|r|r}
\hline\hline
Phenotype      & A      & AB       & B       & O  \\
\hline
Genotype(s)    & aa\ ao & ab       & bb\ bo  & oo \\
No.\ in sample & $N_A$  & $N_{AB}$ & $N_{B}$ & $N_O$ \\
\hline
\end{tabular}
\end{center}
}

\myslide{
\heading{Expected phenotype numbers and allele frequencies}
\begin{eqnarray*}
N_{aa} &=& n_A \left({p_a^2 \over p_a^2 + 2p_ap_o}\right) \\
N_{ao} &=& n_A \left({2p_ap_o \over p_a^2 + 2p_ap_o}\right) \\
N_{bb} &=& n_B \left({p_b^2 \over p_b^2 + 2p_bp_o}\right) \\
N_{bo} &=& n_B \left({2p_bp_o \over p_b^2 + 2p_bp_o}\right)
\end{eqnarray*}
\begin{eqnarray*}
p_a &=& {2N_{aa} + N_{ao} + N_{ab} \over 2N} \\
p_b &=& {2N_{bb} + N_{bo} + N_{ab} \over 2N} \\
p_o &=& {2N_{oo} + N_{ao} + N_{bo} \over 2N}
\end{eqnarray*}
}

\myslide{
\heading{Multinomial probability}
\[
\mbox{P}(K_1=k_1, K_2=k_2, K_3=k_3|p_1,p_2,p_3) = {N \choose k_1 k_2 k_3}p_1^{k_1}p_2^{k_2}p_3^{k_3}
\]
\vfil
For ABO
\[
{N \choose N_A N_{AB} N_B N_O}
\left(p_a^2 + 2p_ap_o\right)^{N_A}
2p_ap_b^{N_{AB}}
\left(p_b^2 + 2p_bp_o\right)^{N_B}
\left(p_o^2\right)^{N_O}
\]
\vfill
No simple formula that will allow you to calculate $p_a$, $p_b$, and
$p_o$.
}

\myslide{
\heading{Mating table with complete self-fertilization}
\begin{center}
\begin{tabular}{ccccc}
\hline\hline
&&\multicolumn{3}{c}{Offsrping genotype} \\
Mating & frequency & $A_1A_1$ & $A_1A_2$ & $A_2A_2$ \\
\hline
$A_1A_1 \times A_1A_1$ & $x_{11}$ & 1 & 0 & 0 \\
$A_1A_2 \times A_1A_2$ & $x_{12}$ & $\fourth$ & $\half$ & $\fourth$ \\
$A_2A_2 \times A_2A_2$ & $x_{22}$ & 0 & 0 & 1 \\
\hline
\end{tabular}
\end{center}
\begin{eqnarray*}
x_{11}' &=& x_{11} + x_{12}/4 \\
x_{12}' &=& x_{12}/2 \\
x_{22}' &=& x_{22} + x_{12}/4 \\
\end{eqnarray*}
}

\myslide{
\heading{Offspring genotype and allele frequencies}
\begin{eqnarray*}
x_{11}' &=& x_{11} + x_{12}/4 \\
x_{12}' &=& x_{12}/2 \\
x_{22}' &=& x_{22} + x_{12}/4 \\
\end{eqnarray*}
\vfil
\begin{eqnarray*}
p' &=& x_{11}' + x_{12}'/2 \\
   &=& x_{11} + x_{12}/4 + x_{12} /4 \\
   &=& x_{11} + x_{12}/2 \\
   &=& p
\end{eqnarray*}
}

\myslide{
\heading{Partial self-fertilization}
\begin{eqnarray*}
x_{11}' &=& p^2(1-\sigma) + (x_{11} + x_{12}/4)\sigma \\
x_{12}' &=& 2pq(1-\sigma) + (x_{12}/2)\sigma  \label{eq:het} \\
x_{22}' &=& q^2(1-\sigma) + (x_{22} + x_{12}/4)\sigma
\end{eqnarray*}
\begin{eqnarray*}
\hat x_{12} &=& 2pq(1-\sigma) + (\hat x_{12}/2)\sigma \\
\hat x_{12}(1 - \sigma/2) &=& 2pq(1-\sigma) \\
\hat x_{12} &=& \frac{2pq(1-\sigma)}{(1-\sigma/2)}
\end{eqnarray*}
\begin{eqnarray*}
\hat x_{11} &=& p^2 + fpq \\
\hat x_{12} &=& 2pq(1-f) \\
\hat x_{22} &=& q^2 + fpq
\end{eqnarray*}
}

\myslide{
\heading{Identity by descent}
\begin{center}
\begin{tabular}{rcccc}
      &            & $A_1$ & $\rightarrow$ & $A_1$ \\
      & $\nearrow$ &       &               & \\
$A_1$ &            &       &               & \\
      & $\searrow$ &       &               & \\
      &            & $A_1$ & $\rightarrow$ & $A_1$
\end{tabular}
\end{center}
}

\myslide{
\heading{Identity by type}
\begin{center}
\begin{tabular}{rcccc}
      &            & $A_1$ & $\rightarrow$ & $A_1$ \\
      & $\nearrow$ &       &               & \\
$A_1$ &            &       &               & \\
      & $\searrow$ &       &               & \\
      &            & $A_2$ & $\rightarrow$ & $A_1$ \\
      & $\uparrow$ &       & $\uparrow$    & \\
      & mutation   &       & mutation      &
\end{tabular}
\end{center}
}

\myslide{
\heading{Genotype frequencies with inbreeding}
\begin{eqnarray*}
x_{11} &=& p^2(1-f) + fp \\
x_{12} &=& 2pq(1-f) \\
x_{22} &=& q^2(1-f) + fq
\end{eqnarray*}
\begin{eqnarray*}
x_{11} &=& p^2 + fpq \\
x_{12} &=& 2pq(1-f) \\
x_{22} &=& q^2 + fpq
\end{eqnarray*}
}


\end{document}

