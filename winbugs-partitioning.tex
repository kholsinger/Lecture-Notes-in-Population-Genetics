\documentclass[12pt]{article}
\usepackage{lecture}
\usepackage{graphics}
\usepackage{html}
\usepackage{url}

\newcommand{\Var}{\mbox{Var}}
\newcommand{\Cov}{\mbox{Cov}}

\newcommand{\copyrightYears}{2001-2008}

\title{Partitioning variance with {\tt WinBUGS}}

\begin{document}

\maketitle

\thispagestyle{first}

As you have already guessed, there is a way to partition variance with
WinBUGS. One of the advantages of using WinBUGS for variance
partitioning is that it is easy to get direct estiamtes of both the
observational and the causal components of variance in a quantitative
genetics design.\footnote{I'll only describe the observational
  component here. Getting the causal components will be left as an
  exercise to be completed as part of Problem \#4. The source code and
  data used for this example is available at
  \htmladdnormallink{http://darwin.eeb.uconn.edu/eeb348/supplements-2006/partitioning-example.txt}{http://darwin.eeb.uconn.edu/eeb348/supplements-2006/partitioning-example.txt}.}

The data has a fairly straightforward structure, although it will look
different from other WinBUGS data you've~seen:

\begin{verbatim}
   sire[] dam[] weight[]
   1 1 99.2789343018915
   1 1 92.8290811609914
   1 1 92.1977611557311
   1 1 96.2203721102781
   1 1 97.435349690979
   1 1 99.5683756647001
   1 1 92.2818423839868
   1 1 99.5110665900414
   1 2 103.230303160319
   .    .     .
   .    .     .
   .    .     .
   7 52 102.100956942292
   7 53 117.842216117441
   7 53 124.174111459722
   7 53 115.02591859104
   7 53 111.885901553834
   7 53 115.320308219732
\end{verbatim}

The first column is the number of the sire involved in the mating, the
second column is the number of the dam involved in the mating, and the
last column is the weight of an individual offspring.

Recall that our objective is to describe how much of the overall
variation is due to weight differences among individuals that share
the same mother, how much is due to differences among mothers in the
average weight of their offspring, and how much is due to differences
among fathers in the average weight of their offspring. One way of
approaching this is to imagine that the weight of each individual is
determined as~follows:
\[
\mbox{weight}_i = \mbox{mean}(\mbox{weight})
                  + \mbox{sire contribution}
                  + \mbox{dam contribution}
                  + \mbox{error} \quad ,
\]
where the sire contribution, the dam contribution, and $\hbox{error}$
are normally distribtued random variables with mean $0$ and variances
of $V_s$, $V_d$, and $V_w$, respectively. We can translate that into
WinBUGS code like~this:

\begin{verbatim}
  # offspring
  for (i in 1:300) {
     weight[i] ~ dnorm(mu[i], tau.within)
     mu[i] <- nu + alpha[sire[i]] + beta[dam[i]]
  }

  # sires
  for (i in 1:6) {
     alpha[i] ~ dnorm(0.0, tau.sire)
  }
  alpha[7] <- -sum(alpha[1:6])

  # dams
  for (i in 1:52) {
     beta[i] ~ dnorm(0.0, tau.dam)
  }
  beta[53] <- -sum(beta[1:52])

  # precisions
  tau.sire <- 1.0/v.sire
  tau.dam <- 1.0/v.dam
  tau.within <- 1.0/v.within
\end{verbatim}

Why the funny precisions, {\tt tau.sire}, {\tt tau.dam}, and {\tt
  tau.within}? It turns out that it was easier for the people who
wrote WinBUGS to describe normal distributions in terms of a mean and
precision, where precision $=$ 1/variance, than in terms of a mean and
variance. Once we know that we need to work with precisions, then all
we need to do is to put appropriate priors on {\tt nu}, {\tt
  v.within}, {\tt v.dam}, and {\tt v.sire}. The most convenient way to
put priors on the variance components is to put a uniform distribution
on the corresponding standard deviations.

\begin{verbatim}
  # priors
  nu ~ dnorm(0, 0.001)
  sd.sire ~ dunif(0, 10)
  sd.dam ~ dunif(0, 10)
  sd.within ~ dunif(0, 10)

  # observational components
  v.sire   <- sd.sire*sd.sire
  v.dam    <- sd.dam*sd.dam
  v.within <- sd.within*sd.within
\end{verbatim}

If you put this all together in a WinBUGS model, then
WinBUGS will produces results like those shown in
Table~\ref{table:results}.\footnote{I would recommend using 10,000
iterations at each step of the analysis. This model takes a bit
longer to settle down than the other ones we've seen.}

\begin{table}
\begin{center}
\begin{tabular}{lc}
\hline\hline
Variance component & Estimate (2.5\%, 97.5\%) \\
\hline
v.sire   & 8.18 (1.306, 31.29) \\
v.dam    & 4.571 (2.281, 7.974) \\
v.within & 11.48 (9.599, 13.68) \\
\hline
\end{tabular}
\end{center}
\caption{Estimated variance components from the full-sib
data.}\label{table:results}
\end{table}

\ccLicense

\end{document}
