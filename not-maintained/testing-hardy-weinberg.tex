\documentclass[12pt]{article}
\usepackage{lecture}
\usepackage{html}
\usepackage{graphics}
\usepackage{epstopdf}

\newcommand{\copyrightYears}{2001-2015}

\title{Testing Hardy-Weinberg}

\begin{document}

\maketitle

\thispagestyle{first}

\section*{Introduction}

Because the Hardy-Weinberg principle tells us what to expect
concerning the genetic composition of a sample when no evolutionary
forces are operating, one of the first questions population
geneticists often ask is ``Are the genotypes in this sample present in
the expected, i.e., Hardy-Weinberg, proportions?'' We ask that
question because we know that if the genotypes are {\it not\/} in
Hardy-Weinberg proportions, at least one of the assumptions underlying
derivation of the principle has been violated, i.e., that there is
some evolutionary force operating on the population, and we know that
we can use the magnitude and direction of the departure to say
something about what those forces might be. In particular, we now know
that inbreeding leads to a deficiency of heterozygotes, and we know
that the extent of that deficiency can be measured by
$f$.\footnote{Quiz question: Which definition of $f$ is relevant for
  determining whether there is a deficiency of heterozygotes?}

What we haven't talked about is (a) how to estimate $f$ from data and
(b) how to tell whether we have good evidence that the estimate is
positive (meaning that there's a deficiency of heterozygotes in the
population) or negative. Both (a) and (b) pose more of a challenge
than you might initially think. After all we also know that the
numbers in our sample may differ from expectation just because of
random sampling error. For example, Table~\ref{table:MN-data} presents
data from a sample of 1000 English blood donors scored for MN
phenotype. M and N are co-dominant, so that heterozygotes can be
distinguished from the two homozygotes. Clearly the observed and
expected numbers don't look very different.\footnote{For the time
  being, I simply calculated the expected numbers in the way you'd
  tell your students in introductory biology to do it: (1) Use the
  sample frequency of M to estimate its population frequency. (This is
  a maximum-likelihood estimate, by the way. (2) Calculate the
  expected frequency of each genotype from the Hardy-Weinberg
  proportions. (3) Calculated the expected numbers of each genotype by
  multiplying the expected frequency of each by the total sample
  size.} The differences semm likely to be attributable purely to
chance, but we need some way of assessing that
``likeliness.''\index{testing Hardy-Weinberg}

\begin{table}
\begin{center}
\begin{tabular}{cccc}
\hline\hline
          &          & Observed & Expected \\
Phenotype & Genotype & Number   & Number   \\
\hline
M         & mm       & 298      & 294.3 \\
MN        & mn       & 489      & 496.3 \\
N         & nn       & 213      & 209.3 \\
\hline
\end{tabular}
\end{center}
\caption{Adapted from Table 2.4 in~\cite{Hedrick-2000}
  (from~\cite{Cleghorn-1960})}\label{table:MN-data}
\end{table}

\section*{Testing Hardy-Weinberg}

One approach to testing the hypothesis that genotypes are in
Hardy-Weinberg proportions is quite simple. We can simply do a
$\chi^2$ or $G$-test for goodness of fit between observed and
predicted genotype (or phenotype) frequencies, where the predicted
genotype frequencies are derived from our estimates of the allele
frequencies in the population.\footnote{If you're not familiar with
  the $\chi^2$ or $G$-test for goodness of fit, consult any
  introductory statistics or biostatistics book, and you'll find a
  description. In fact, you probably don't have to go that far. You
  can probably find one in your old genetics textbook. Or you can just
  boot up your browser and head to Wikipedia:
  \myurl{http://en.wikipedia.org/wiki/Goodness\_of\_fit}. }\index{testing
  Hardy-Weinberg!goodness of fit} There's only one problem. To do
either of these tests we have to know how many degrees of freedom are
associated with the test. How do we figure that out? In general, the
formula is
\begin{eqnarray*}
\hbox{d.f.} &=& (\hbox{\# of categories in the data -1 }) \\
&&- (\hbox{\#
              number of parameters estimated from the data}) \quad .
\end{eqnarray*}
For this problem we have
\begin{eqnarray*}
\hbox{d.f.} &=& (\hbox{\# of phenotype categories in the data - 1}) \\
&&- (\hbox{\# of allele frequencies estimated from the data}) \\
&=& (3-1) - 1 \\
&=& 1 \quad .
\end{eqnarray*}
In the ABO blood group we have 4 phenotype categories, and 3
allele frequencies. That means that a test of whether a particular
data set has genotypes in Hardy-Weinberg proportions will have
$(4-1)-(3-1) = 1$ degrees of freedom for the test. Notice that this
also means that if you have completely dominant markers, like RAPDs or
AFLPs, you can't determine whether genotypes are in Hardy-Weinberg
proportions because you have 0 degrees of freedom available for the
test.

\subsection*{An example}

Table~\ref{table:abo-data} exhibits data drawn from a study of
phenotypic variation among individuals at the ABO blood locus:

\begin{table}
\begin{center}
\begin{tabular}{lccccc}
Phenotype &   A &  AB &   B &   O & Total \\
Observed  & 862 & 131 & 365 & 702 & 2060
\end{tabular}
\end{center}
\caption{Data on variation in ABO blood type.}\label{table:abo-data}
\end{table}
The maximum-likelihood estimate of allele frequencies,
assuming Hardy-Weinberg, is:\footnote{Take my word for it, or run the
  EM algorithm on these data yourself.}
\begin{eqnarray*}
p_a &=& 0.281 \\
p_b &=& 0.129 \\
p_o &=& 0.590 \quad , \\
\end{eqnarray*}
giving expected numbers of 846, 150, 348, and 716 for the
four phenotypes. $\chi^2_1 = 3.8$, $0.05 < p < 0.1$.

\section*{A Bayesian approach}

We've already seen how to use JAGS to provide allele frequency
estimates from phenotypic data at the \htmladdnormallink{ABO
  locus}{http://darwin.eeb.uconn.edu/eeb348/supplements-2014/multinomial.txt}.\index{testing Hardy-Weinberg!Bayesian approach}
\begin{verbatim}
model {
   # likelihood
   pi[1] <- p.a*p.a + 2*p.a*p.o
   pi[2] <- 2*p.a*p.b
   pi[3] <- p.b*p.b + 2*p.b*p.o
   pi[4] <- p.o*p.o
   x[1:4] ~ dmulti(pi[],n)

   # priors
   a1 ~ dexp(1)
   b1 ~ dexp(1)
   o1 ~ dexp(1)
   p.a <- a1/(a1 + b1 + o1)
   p.b <- b1/(a1 + b1 + o1)
   p.o <- o1/(a1 + b1 + o1)

   n <- sum(x[])
}
list(x=c(862, 131, 365, 702))
\end{verbatim}
As you may recall, this produced the following results:
\begin{verbatim}
> source("multinomial.R")
Compiling model graph
   Resolving undeclared variables
   Allocating nodes
   Graph Size: 20

Initializing model

  |++++++++++++++++++++++++++++++++++++++++++++++++++| 100%
  |**************************************************| 100%
Inference for Bugs model at "multinomial.txt", fit using jags,
 5 chains, each with 2000 iterations (first 1000 discarded)
 n.sims = 5000 iterations saved
         mu.vect sd.vect   2.5%    25%    50%    75%  97.5%  Rhat n.eff
p.a        0.282   0.008  0.266  0.276  0.282  0.287  0.297 1.001  5000
p.b        0.129   0.005  0.118  0.125  0.129  0.133  0.140 1.001  5000
p.o        0.589   0.008  0.573  0.584  0.589  0.595  0.606 1.001  5000
deviance  27.811   2.007 25.830 26.363 27.229 28.577 33.245 1.001  4400

For each parameter, n.eff is a crude measure of effective sample size,
and Rhat is the potential scale reduction factor (at convergence, Rhat=1).

DIC info (using the rule, pD = var(deviance)/2)
pD = 2.0 and DIC = 29.8
DIC is an estimate of expected predictive error (lower deviance is better).
>
\end{verbatim}
Now that we know about inbreeding coefficients and that they allow us
to measure the departure of genotype frequencies from Hardy-Weinberg
proportions, we can modify this a bit and estimate allele frequencies
without assuming that genotypes are in Hardy-Weinberg proportions.
\begin{verbatim}
model {
   # likelihood
   pi[1] <- p.a*p.a + f*p.a*(1-p.a) + 2*p.a*p.o*(1-f)
   pi[2] <- 2*p.a*p.b*(1-f)
   pi[3] <- p.b*p.b + f*p.b*(1-p.b) + 2*p.b*p.o*(1-f)
   pi[4] <- p.o*p.o + f*p.o*(1-p.o)
   x[1:4] ~ dmulti(pi[],n)

   # priors
   a1 ~ dexp(1)
   b1 ~ dexp(1)
   o1 ~ dexp(1)
   p.a <- a1/(a1 + b1 + o1)
   p.b <- b1/(a1 + b1 + o1)
   p.o <- o1/(a1 + b1 + o1)

   f ~ dunif(0,1)

   n <- sum(x[])
}
\end{verbatim}
This simple change produces the following results:
\begin{verbatim}
> source("abo-inbreeding.R")
Compiling model graph
   Resolving undeclared variables
   Allocating nodes
   Graph Size: 30

Initializing model

  |++++++++++++++++++++++++++++++++++++++++++++++++++| 100%
  |**************************************************| 100%
Inference for Bugs model at "abo-inbreeding.txt", fit using jags,
 5 chains, each with 2000 iterations (first 1000 discarded)
 n.sims = 5000 iterations saved
         mu.vect sd.vect   2.5%    25%    50%    75%  97.5%  Rhat n.eff
f          0.403   0.139  0.059  0.326  0.429  0.505  0.599 1.013   550
p.a        0.349   0.027  0.290  0.332  0.352  0.368  0.392 1.006   960
p.b        0.161   0.014  0.132  0.152  0.162  0.171  0.186 1.006   840
p.o        0.490   0.039  0.429  0.461  0.485  0.514  0.577 1.006  1000
deviance  25.200   2.416 22.249 23.411 24.716 26.342 31.206 1.007   470

For each parameter, n.eff is a crude measure of effective sample size,
and Rhat is the potential scale reduction factor (at convergence, Rhat=1).

DIC info (using the rule, pD = var(deviance)/2)
pD = 2.9 and DIC = 28.1
DIC is an estimate of expected predictive error (lower deviance is better).
>
\end{verbatim}
Notice that the allele frequency estimates have changed quite a bit
and that the posterior mean of $f$ is about 0.40. On first appearance,
that would seem to indicate that we have lots of inbreeding in this
sample. {\bf BUT} it's a human population. It doesn't seem very
likely that a human population is really that highly inbred.

Indeed, take a closer look at {\it all\/} of the information we have
about that estimate of $f$. The 95\% credible interval for $f$ is
between 0.06 and 0.60. That suggests that we don't have much
information at all about $f$ from these data.\footnote{That shouldn't
  be too surprising, since any information we have about $f$ comes
  indirectly through our allele frequency estimates.} How can we tell
if the model with inbreeding is better than the model that assumes
genotypes are in Hardy-Weinberg proportions?

\subsection*{The Deviance Information Criterion}

A widely used statistic for comparing models in a Bayesian framework
is the Deviance Information Criterion.\index{Deviance Information
  Criterion} {\tt R2jags} calculates an estimate of it for us
automatically, but you need to know that if you're serious about model
comparison, yous shouldn't rely on the DIC calculation from {\tt
  R2jags} unless you've verified it.\footnote{If you're interested in
  learning more, feel free to ask, but I'm afraid both the explanation
  and the solution are a little complicated.} Fortunately, in this
case, the results are fairly reliable.\footnote{You'll just have to
  trust me on this unless you asked the last question.} The
results of the DIC calculations for our two models are summarized in
Table~\ref{table:ABO-dic}.

\begin{table}
\begin{center}
\begin{tabular}{c|ccc}
\hline\hline
Model   & deviance & pD    & DIC \\
\hline
$f > 0$ & 25.2 & 2.9 & 28.1 \\
$f = 0$ & 27.8 & 2.0 & 29.9 \\
\hline
\end{tabular}
\end{center}
\caption{DIC calculations for the ABO example.}\label{table:ABO-dic}
\end{table}

The {\tt deviance} is a measure of how well the model fits the data,
specifically -2 times the average of the log likelihood values
calculated from the parameters in each sample from the posterior. {\tt
  pD} is a measure of model complexity, roughly speaking the number of
parameters in the model.\footnote{Notice that we estimated 2
  parameters in the $f=0$ model (2 allele frequencies) and 3
  parameters in the $f>0$ model (2 allele frequencies plus the
  inbreeding coefficient).} DIC is a composite measure of how well the
model does. It's a compromise between fit and complexity, and smaller
DICs are preferred. A difference of more than 7-10 units is regarded
as strong evidence in favor of the model with the smaller DIC.

In this case the difference in DIC values is only about 0.8, so we
have very little evidence for $f > 0$ model for these data. This is
consistent with the weak evidence for a departure from Hardy-Weinberg
that was revealed in the $\chi^2$ analysis.

\bibliography{popgen}
\bibliographystyle{plain}

\ccLicense

\end{document}
